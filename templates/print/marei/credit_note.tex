\documentclass[paper=a4,fontsize=10pt]{scrartcl}
\usepackage{kiviletter}


% Variablen, die in settings verwendet werden
\newcommand{\lxlangcode} {<%template_meta.language.template_code%>}
\newcommand{\lxmedia} {<%media%>}
\newcommand{\lxcurrency} {<%currency%>}
\newcommand{\kivicompany} {<%employee_company%>}

% settings: Einstellungen, Logo, Briefpapier, Kopfzeile, Fusszeile
\input{insettings.tex}


% laufende Kopfzeile:
\ourhead{\kundennummer}{<%customernumber%>}{\gutschrift}{<%invnumber%>}{<%invdate%>}


\begin{document}
	

<%if shiptoname%>
\makeatletter
\begin{lrbox}\shippingAddressBox
	\parbox{\useplength{toaddrwidth}}{
		\backaddr@format{\scriptsize\usekomafont{backaddress}%
			\strut abweichende Lieferadresse
		}
		\par\smallskip
		\setlength{\parskip}{\z@}
		\par
		\normalsize
		<%shiptoname%>\par
		<%if shiptocontact%> <%shiptocontact%><%end if%>\par
		<%shiptodepartment_1%>\par
		<%shiptodepartment_2%>\par
		<%shiptostreet%>\par
		<%shiptozipcode%> <%shiptocity%>
	}
\end{lrbox}
\makeatother
<%end if%>

	
\setkomavar{title}{
	\gutschrift~
	\nr ~<%invnumber%>
}

<%if invnumber_for_credit_note%>
\setkomavar*{myref}{\fuerRechnung}
\setkomavar{myref}{<%invnumber_for_credit_note%>}
<%end if%>

\setkomavar*{date}{\datum}

\setkomavar{date}{<%transdate%>}
\setkomavar{customer}{<%customernumber%>}
\setkomavar{fromname}{<%employee_name%>}
\setkomavar{fromphone}{<%employee_tel%>}
\setkomavar{fromemail}{<%employee_email%>}
	
\begin{letter}{
		<%name%>\strut\\
		<%if department_1%><%department_1%>\\<%end if%>
		<%if department_2%><%department_2%>\\<%end if%>
		<%cp_givenname%> <%cp_name%>\strut\\
		<%street%>\strut\\
		<%zipcode%> <%city%>\strut\\
		<%country%> \strut
	}

% Bei Kontaktperson Anrede nach Geschlecht unterscheiden.
% Bei natürlichen Personen persönliche Anrede, sonst allgemeine Anrede.
\opening{
	\ifstr{<%cp_name%>}{}
		{<%if natural_person%><%greeting%> <%name%>,<%else%>\anrede<%end if%>}
		{
			\ifstr{<%cp_gender%>}{f}
				{\anredefrau}
				{\anredeherr}
				<%cp_title%> <%cp_name%>,
			}
		}
\thispagestyle{kivitendo.letter.first}


\gutschriftformel

\begin{PricingTabular*}
% eigentliche Tabelle
\FakeTable{
	<%foreach number%>%
	<%runningnumber%> &%
	<%number%> &%
	\textbf{<%description%>}%
		<%if longdescription%>\ExtraDescription{<%longdescription%>}<%end longdescription%>%
		<%if serialnumber%>\ExtraDescription{\seriennummer: <%serialnumber%>}<%end serialnumber%>%
		<%if ean%>\ExtraDescription{\ean: <%ean%>}<%end ean%>%
		<%if projectnumber%>\ExtraDescription{\projektnummer: <%projectnumber%>}<%end projectnumber%>%
		&%
		<%qty%> <%unit%> &%
		<%sellprice%>&%
		\ifstr{<%p_discount%>}{0}{}{\sffamily\scriptsize{(-<%p_discount%>\,\%)}}%
			<%linetotal%>\tabularnewline
			<%end number%>
	}
	\begin{PricingTotal}
		% Tabellenende letzte Seite
		\nettobetrag & <%subtotal%>\\
		<%foreach tax%>
		<%taxdescription%> & <%tax%>\\
		<%end tax%>
		\bfseries\schlussbetrag &  \bfseries <%ordtotal%>\\
	\end{PricingTotal}
\end{PricingTabular*}

<%if notes%>
<%notes%>
\medskip
<%end if%>

\closing{\gruesse}

\end{letter}

\end{document}
