\documentclass[paper=a4,fontsize=10pt]{scrartcl}
\usepackage{kiviletter}


% Variablen, die in settings verwendet werden
\newcommand{\lxlangcode} {<%template_meta.language.template_code%>}
\newcommand{\lxmedia} {<%media%>}
\newcommand{\lxcurrency} {<%currency%>}
\newcommand{\kivicompany} {<%employee_company%>}

% settings: Einstellungen, Logo, Briefpapier, Kopfzeile, Fusszeile
\input{insettings.tex}


% laufende Kopfzeile:
\ourhead{\kundennummer}{<%customernumber%>}{\rechnung}{<%invnumber%>}{<%invdate%>}


\begin{document}

\setkomavar{title}{
	\rechnung~
	\nr ~<%invnumber%>
}
\setkomavar*{date}{\rechnungsdatum}
\setkomavar{date}{<%invdate%>}
\setkomavar*{myref}{\mahnung~\nr}
\setkomavar{myref}{<%dunning_id%>}
\setkomavar{customer}{<%customernumber%>}
\setkomavar{fromname}{<%employee_name%>}
\setkomavar{fromphone}{<%employee_tel%>}
\setkomavar{fromemail}{<%employee_email%>}
<%if globalprojectnumber%>
	\setkomavar{projectID}{<%globalprojectnumber%>}
<%end globalprojectnumber%>
\setkomavar{transaction}{<%transaction_description%>}

\begin{letter}{
		<%name%>\strut\\
		<%if department_1%><%department_1%>\\<%end if%>
		<%if department_2%><%department_2%>\\<%end if%>
		<%cp_givenname%> <%cp_name%>\strut\\
		<%street%>\strut\\
		<%zipcode%> <%city%>\strut\\
		<%country%> \strut
	}

% Bei Kontaktperson Anrede nach Geschlecht unterscheiden.
% Bei natürlichen Personen persönliche Anrede, sonst allgemeine Anrede.
\opening{
	\ifstr{<%cp_name%>}{}
		{<%if natural_person%><%greeting%> <%name%>,<%else%>\anrede<%end if%>}
		{
			\ifstr{<%cp_gender%>}{f}
				{\anredefrau}
				{\anredeherr}
				<%cp_title%> <%cp_name%>,
		}
}
\thispagestyle{kivitendo.letter.first}

\mahnungsrechnungsformel



\setlength\LTleft\parindent     % Tabelle beginnt am linken Textrand
\setlength\LTright{0pt}         % Tabelle endet am rechten Textrand
\begin{longtable}{@{}p{7cm}@{\extracolsep{\fill}}r@{}}
% Tabellenkopf
\hline
\textbf{\posten} & \textbf{\betrag} \\
\hline\\
\endhead

% Tabellenkopf erste Seite
\hline
\textbf{\posten} & \textbf{\betrag} \\
\hline\\[-0.5em]
\endfirsthead

% Tabellenende
\\
\multicolumn{2}{@{}r@{}}{\weiteraufnaechsterseite}
\endfoot

% Tabellenende letzte Seite
\hline\\
\multicolumn{1}{@{}l}{\schlussbetrag} & <%invamount%> \currency\\
\hline\hline\\
\endlastfoot

% eigentliche Tabelle
Mahngebühren & <%fee%> \currency \\
Zinsen & <%interest%> \currency \\
\\[-0.8em]

\end{longtable}


\vspace{0.2cm}

\bitteZahlenBis~<%duedate%>.

\closing{\gruesse}

\end{letter}

\end{document}
