\input{inheaders.tex}


% Variablen, die in settings verwendet werden
\newcommand{\lxlangcode} {<%template_meta.language.template_code%>}
\newcommand{\lxmedia} {<%media%>}
\newcommand{\lxcurrency} {<%currency%>}
\newcommand{\kivicompany} {<%employee_company%>}

% settings: Einstellungen, Logo, Briefpapier, Kopfzeile, Fusszeile
\input{insettings.tex}


% laufende Kopfzeile:
\ourhead{}{}{\sammelrechnung}{}{}


\begin{document}

\ourfont

\begin{minipage}[t]{8cm}
  \vspace*{1.0cm}

  <%name%>

  <%street%>

  ~

  <%zipcode%> <%city%>

  <%country%>
\end{minipage}
\hfill
\begin{minipage}[t]{6cm}
  \hfill{\LARGE\textbf{\sammelrechnung}}

  \vspace*{0.2cm}

  \ansprechpartner:\hfill <%employee_name%>
\end{minipage}

\vspace*{1.5cm}

\hfill

% Bei Kontaktperson Anrede nach Geschlecht unterscheiden.
% Bei natürlichen Personen persönliche Anrede, sonst allgemeine Anrede.
\ifthenelse{\equal{<%cp_name%>}{}}{
  <%if natural_person%><%greeting%> <%name%>,<%else%>\anrede<%end if%>}{
  \ifthenelse{\equal{<%cp_gender%>}{f}}
    {\anredefrau}{\anredeherr} <%cp_title%> <%cp_name%>,}\\

\sammelrechnungsformel\\

\vspace{0.5cm}


%
% - longtable kann innerhalb der Tabelle umbrechen
% - da der Umbruch nicht von Lx-Office kontrolliert wird, kann man keinen
%   Übertrag machen
%
\setlength\LTleft\parindent     % Tabelle beginnt am linken Textrand
\setlength\LTright{0pt}         % Tabelle endet am rechten Textrand
\begin{longtable}{@{\extracolsep{\fill}}rrrrrrr@{}}
% Tabellenkopf
\hline
\textbf{\rechnung~\nr} & \textbf{\datum} & \textbf{\faellig} &
\textbf{\aktuell} & \textbf{\asDreissig} & \textbf{\asSechzig} & \textbf{\asNeunzig}\\
\hline\\
\endhead

% Tabellenkopf erste Seite
\hline
\textbf{\rechnung~\nr} & \textbf{\datum} & \textbf{\faellig} &
\textbf{\aktuell} & \textbf{\asDreissig} & \textbf{\asSechzig} & \textbf{\asNeunzig}\\
\hline\\[-0.5em]
\endfirsthead

% Tabellenende
\\
\multicolumn{7}{@{}r@{}}{\weiteraufnaechsterseite}
\endfoot

% Tabellenende letzte Seite
\hline\\
\multicolumn{3}{@{}l}{\textbf{\zwischensumme}} & \textbf{<%c0total%>} & \textbf{<%c30total%>} & \textbf{<%c60total%>} & \textbf{<%c90total%>}\\
\hline\\
\multicolumn{6}{@{}l}{\textbf{\schlussbetrag}} & \textbf{<%total%>} \\
\hline\hline\\
\endlastfoot

% eigentliche Tabelle
<%foreach invnumber%>
          <%invnumber%> & <%invdate%> & <%duedate%> &
          <%c0%> & <%c30%> & <%c60%> & <%c90%> \\
<%end invnumber%>

\end{longtable}


\vspace{0.2cm}

\gruesse \\ \\ \\
  <%employee_name%>

\end{document}

