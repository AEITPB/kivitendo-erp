\input{inheaders.tex}


% Variablen, die in settings verwendet werden
\newcommand{\lxlangcode} {<%template_meta.language.template_code%>}
\newcommand{\lxmedia} {<%media%>}
\newcommand{\lxcurrency} {<%currency%>}
\newcommand{\kivicompany} {<%employee_company%>}

% settings: Einstellungen, Logo, Briefpapier, Kopfzeile, Fusszeile
\input{insettings.tex}


% laufende Kopfzeile:
\ourhead{}{}{\bestellung}{<%ordnumber%>}{<%orddate%>}


\begin{document}

\ourfont
\begin{minipage}[t]{8cm}
  \scriptsize

  {\color{gray}\underline{\firma\ $\cdot$ \strasse\ $\cdot$ \ort}}
  \normalsize

  \vspace*{0.3cm}

  <%name%>

  <%if department_1%><%department_1%><%end if%>

  <%if department_2%><%department_2%><%end if%>

  <%cp_givenname%> <%cp_name%>

  <%street%>

  ~

  <%zipcode%> <%city%>

  <%country%>
\end{minipage}
\hfill
\begin{minipage}{6cm}
  \rightline{\LARGE\textbf{\textit{\bestellung}}} \vspace*{0.2cm}
  \rightline{\large\textbf{\textit{\nr ~<%ordnumber%>%
  }}} \vspace*{0.2cm}

  \datum:\hfill <%orddate%>

  <%if cusordnumber%>\unsereBestellnummer:\hfill <%cusordnumber%><%end if%>

  <%if quonumber%>\angebot~\nr:\hfill <%quonumber%><%end if%>

  \ansprechpartner:\hfill <%employee_name%>

  \textTelefon \hfill <%employee_tel%>

  \textEmail \hfill <%employee_email%>
\end{minipage}

<%if shiptoname%>
  \vspace{0.8cm}
  \scriptsize \underline{\abweichendeLieferadresse:}\\
  \normalsize    <%shiptoname%>

                 <%if shiptocontact%> <%shiptocontact%><%end if%>

                 <%shiptodepartment_1%>

                  <%shiptodepartment_2%>

                  <%shiptostreet%>

                  <%shiptozipcode%> <%shiptocity%>
<%end if%>
\vspace*{1.5cm}

\hfill

% Bei Kontaktperson Anrede nach Geschlecht unterscheiden.
% Bei natürlichen Personen persönliche Anrede, sonst allgemeine Anrede.
\ifthenelse{\equal{<%cp_name%>}{}}{
  <%if natural_person%><%greeting%> <%name%>,<%else%>\anrede<%end if%>}{
  \ifthenelse{\equal{<%cp_gender%>}{f}}
    {\anredefrau}{\anredeherr} <%cp_title%> <%cp_name%>,}\\

\bestellformel\\

\vspace{0.5cm}


%
% - longtable kann innerhalb der Tabelle umbrechen
% - da der Umbruch nicht von Lx-Office kontrolliert wird, kann man keinen
%   Übertrag machen
% - Innerhalb des Langtextes <%longdescription%> wird nicht umgebrochen.
%   Falls das gewünscht ist, \\ mit \renewcommand umschreiben (siehe dazu:
%   http://www.lx-office.org/uploads/media/Lx-Office_Anwendertreffen_LaTeX-Druckvorlagen-31.01.2011_01.pdf)
%
\setlength\LTleft\parindent     % Tabelle beginnt am linken Textrand
\setlength\LTright{0pt}         % Tabelle endet am rechten Textrand
\begin{longtable}{@{}rrp{7cm}@{\extracolsep{\fill}}rr@{}}
% Tabellenkopf
\hline
\textbf{\position} & \textbf{\menge} & \textbf{\bezeichnung} & \textbf{\einzelpreis} & \textbf{\gesamtpreis} \\
\hline\\
\endhead

% Tabellenkopf erste Seite
\hline
\textbf{\position} & \textbf{\menge} & \textbf{\bezeichnung} & \textbf{\einzelpreis} & \textbf{\gesamtpreis} \\
\hline\\[-0.5em]
\endfirsthead

% Tabellenende
\\
\multicolumn{5}{@{}r@{}}{\weiteraufnaechsterseite}
\endfoot

% Tabellenende letzte Seite
\hline\\
\multicolumn{4}{@{}l}{\nettobetrag} & <%subtotal%> \currency\\
<%foreach tax%>
\multicolumn{4}{@{}l}{<%taxdescription%>} & <%tax%> \currency\\
<%end tax%>
\multicolumn{4}{@{}l}{\textbf{\schlussbetrag}} &  \textbf{<%ordtotal%>} \currency\\
\hline\hline\\
\endlastfoot

% eigentliche Tabelle
<%foreach number%>
          <%runningnumber%> &
          \raggedleft <%qty%> <%unit%> &
          \textbf{<%description%>} &
          <%sellprice%> \currency &
          \ifthenelse{\equal{<%p_discount%>}{0}}{}{\sffamily\scriptsize{(-<%p_discount%> \%)}}
          \ourfont{<%linetotal%> \currency} \\*  % kein Umbruch nach der ersten Zeile, damit Beschreibung und Langtext nicht getrennt werden

          <%if longdescription%> && \scriptsize <%longdescription%>\\<%end longdescription%>
          <%if reqdate%> && \scriptsize \lieferdatum: <%reqdate%>\\<%end reqdate%>
          <%if serialnumber%> && \scriptsize \seriennummer: <%serialnumber%>\\<%end serialnumber%>
          <%if ean%> && \scriptsize \ean: <%ean%>\\<%end ean%>
          <%if projectnumber%> && \scriptsize \projektnummer: <%projectnumber%>\\<%end projectnumber%>
          <%if make%>
            <%foreach make%>
              \ifthenelse{\equal{<%make%>}{<%name%>}}{&& \artikelnummer: <%model%>\\}{}
            <%end foreach%>
          <%end if%>
          \\[-0.8em]
<%end number%>

\end{longtable}


\vspace{0.2cm}

<%if notes%>
        \vspace{5mm}
        <%notes%>
        \vspace{5mm}
<%end if%>

<%if delivery_term%>
  \lieferung ~<%delivery_term.description_long%>\\
<%end delivery_term%>

\gruesse \\ \\ \\
  <%employee_name%>

\end{document}
